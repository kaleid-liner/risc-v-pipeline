\documentclass{article}

\usepackage{xeCJK}
\usepackage{fancyhdr}
\usepackage{extramarks}
\usepackage{amsmath}
\usepackage{amsthm}
\usepackage{amsfonts}
\usepackage{tikz}
\usepackage[plain]{algorithm}
\usepackage{algpseudocode}
\usepackage{enumerate}
\usepackage{xcolor, minted}
\usepackage[inline]{enumitem}
\usepackage{booktabs}
\usepackage{makecell}
\usemintedstyle{colorful}
\definecolor{bg}{rgb}{0.95, 0.95, 0.95}
\definecolor{lk}{rgb}{0.5, 0.1, 0.1}
\setminted{bgcolor=bg,
           numbers=right,
           samepage}
\setmonofont{Droid Sans Mono for Powerline}

\usetikzlibrary{automata,positioning}

\usepackage[
  colorlinks,
  urlcolor=lk,
  linkcolor=black,
  citecolor=red,
  ]{hyperref}

\usepackage{tocloft}
\renewcommand\cfttoctitlefont{\huge\bfseries}
\renewcommand\cftsecfont{\Large\bfseries}
\renewcommand\cftsubsecfont{\large}
\renewcommand\cftsubsubsecfont{\large}
\setlength\cftbeforesecskip{20pt}
\setlength\cftbeforesubsecskip{10pt}
\setlength\cftbeforesubsubsecskip{8pt}

\setlength\parskip{0.8em}
\setlength\lineskip{0.5em}


%
% Basic Document Settings
%
\setmainfont[ItalicFont=Times New Roman Italic]{Arial}
\setCJKmainfont[ItalicFont=STKaiti, BoldFont=STHeiti]{STXihei}

\topmargin=-0.45in
\evensidemargin=0in
\oddsidemargin=0in
\textwidth=6.5in
\textheight=9.0in
\headsep=0.25in

\linespread{1.1}

\pagestyle{fancy}
\lhead{\reportAuthorName}
\chead{\reportMainTitle}
\rhead{\leftmark}
\cfoot{\thepage}

\renewcommand\headrulewidth{0.4pt}
\renewcommand\footrulewidth{0.4pt}

\setlength\parindent{0pt}

%
% Create Problem Sections
%


%
% Homework Details
%   - Title
%   - Due date
%   - Class
%   - Section/Time
%   - Instructor
%   - Author
%

\newcommand{\reportMainTitle}{Implementation of a LRU/ FIFO Cache}
\newcommand{\reportTitle}{Lab\ \#2}
\newcommand{\reportDueTime}{May 17, 2020 at 23:59pm}
\newcommand{\reportClass}{Computer Architecture}
\newcommand{\reportClassTime}{}
\newcommand{\reportClassInstructor}{周学海}
\newcommand{\reportAuthorName}{魏剑宇}
\newcommand{\reportStudentNo}{PB17111586}

%
% Title Page
%

\title{
  \vspace{2in}
  \textbf{\reportMainTitle}\\
  \vspace{0.2in}
  \Large\textit{\reportClass:\ \reportTitle}\\
  \vspace{0.1in}
  \normalsize\vspace{0.1in}\small{Due\ on\ \reportDueTime}\\
  \vspace{3in}
}

\author{\reportAuthorName \\ \reportStudentNo}
\date{May 17, 2020}


%
% Various Helper Commands
%

% Useful for algorithms
\newcommand{\alg}[1]{\textsc{\bfseries \footnotesize #1}}

% For derivatives
\newcommand{\deriv}[1]{\frac{\mathrm{d}}{\mathrm{d}x} (#1)}

% For partial derivatives
\newcommand{\pderiv}[2]{\frac{\partial}{\partial #1} (#2)}

% Integral dx
\newcommand{\dx}{\mathrm{d}x}

% Alias for the Solution section header

% Probability commands: Expectation, Variance, Covariance, Bias
\newcommand{\E}{\mathrm{E}}
\newcommand{\Var}{\mathrm{Var}}
\newcommand{\Cov}{\mathrm{Cov}}
\newcommand{\Bias}{\mathrm{Bias}}

\begin{document}

\maketitle

\pagebreak

\tableofcontents\label{toc}

\pagebreak

\newpage

\section{实验环境和工具}
\begin{itemize}
  \item
    \textbf{Host OS}: macOS Mojave 10.14.6
  \item
    \textbf{Guest OS}: Ubuntu Bionic Beaver
  \item
    \textbf{VSCode}: 1.44.2
  \item
    \textbf{Vivado}: HL WebPack Edition
  \item
    \textbf{Git}: 2.20.1 (Apple Git-117)
\end{itemize}
本次实验 Simulation 部分在 Ubuntu 虚拟机中完成 (Vivado 没有提供 macOS 版本),代码编写在 mac 上使用 VSCode 完成.

我的做法是在虚拟机 vivado 创建项目时,将源码的文件夹添加进去的时候,不拷贝,而是映射到项目中。然后在 mac 上通过共享文件夹将源码映射到 ubuntu 上,这样,源码编辑和 Git 版本控制都可以在 mac 上完成,只需在仿真的时候切换到 Ubuntu。整个一套 workflow 还是非常流畅舒适的。

\section{Cache 设计}

cache 部分的主要更改是替换策略的更改。其他的除了需要给一些变量添加一个 WAY 的维度外,还有就是需要确定是否有命中明确定命中的是哪一路. 这一点我通过一个 for 语句要实现:
\begin{minted}{verilog}
always @ (*) begin
    cache_hit = 1'b0;
    way_select = 0;
    for (integer i = 0; i < WAY_CNT; i++) begin
        if (valid[set_addr][i] && cache_tags[set_addr][i] == tag_addr) begin
            cache_hit = 1'b1;
            way_select = i;
        end
    end
end
\end{minted}

\subsection{LRU \& FIFO}
\label{impl:lru}
我这里重点描述一下我认为自己设计得非常不错的地方.

像 LRU 或者 FIFO,一种实现思路是用一个位数比较大的寄存器,存储上一次访问到当前的时间. FIFO 的实现还可以用一个队列来存储每次访问.

不过像这样的实现方式事实上有一些问题。姑且不谈上一次的时间每个周期都需要更新,并且可能溢出了位数上限,还有问题就是,每次选出到当前时间最大的时候,需要用到很多的加法器和比较器,会使用很多的资源,实现起来也比较复杂。

而我则使用了一种使用寄存器资源很小,电路资源也很少的实现方式————用一个矩阵来存储历史:
\begin{minted}{verilog}
reg [WAY_CNT-1:0] history_matrix [SET_SIZE][WAY_CNT];
\end{minted}
对于每个 cache 组,\mintinline{text}{history_matrix} 是一个 \mintinline{text}{WAY_CNT x WAY_CNT bit} 大小的矩阵.(这个矩阵非常小,在 \mintinline{text}{WAY_CNT = 4} 的情况下,只有使用历史时间存储的 1/8 大小). 这个矩阵中,$history[i][j] = 1$ 表示的意思是,第 i 路 cache 比 第 j 路 cache 更早使用.

这样我们就可以发现,判断某个 cache 是否被替换就非常容易了:
\begin{minted}{verilog}
&history_matrix[set_addr][i] == 1
\end{minted}
也就是说,这一路 i 对应的行全为 1,通过一个位与操作就可以实现了. 相对于一大堆比较器的版本,可以说是很简洁并且高效了.

更新也比较简单,只需要在每次访问的时候,将该行全设为 0(除了对角线),该列全设为 1.
\begin{minted}{verilog}
for (integer i = 0; i < WAY_CNT; i++)
    history_matrix[set_addr][i][way_select] <= 1'b1;
// set row to 0
for (integer j = 0; j < WAY_CNT; j++)
    if (j != way_select)
        history_matrix[set_addr][way_select][j] <= 1'b0;
end
\end{minted}
这种方式下,LRU 和 FIFO 唯一的区别是 FIFO 只在调入缓存的时候需要更新,LRU 在访问和调入的时候都需要更新.

\subsection{接入 CPU}
为了接入 CPU,主要要做的有亮点:产生两个新的控制信号 \mintinline{text}{rd_req} 和 \mintinline{text}{wr_req}. 这两个信号就是在 cache 中用到的两个信号的意思。其实现也很简单:
\begin{minted}{verilog}
assign rd_req = opcode == `OP_LOAD;
assign wr_req = opcode == `OP_STORE;
\end{minted}
剩下这需要把它们放到流水段寄存器中传过来即可。

主要是需要更改 Hazard 模块。在 Hazard 模块中,需要在有请求且 miss 的时候,产生流水线气泡.
\begin{minted}{verilog}
always @(*) begin
    if ((rd_req | wr_req) & miss) begin
        mem_bubbleF = 1;
        mem_bubbleD = 1;
        mem_bubbleE = 1;
        mem_bubbleM = 1;
        mem_flushW = 1;
    end else begin
        mem_bubbleF = 0;
        mem_bubbleD = 0;
        mem_bubbleE = 0;
        mem_bubbleM = 0;
        mem_flushW = 0;
    end
end
\end{minted}
这里气泡需要插入一直到 MEM 的信号,所以通过 bubble MEM 之前的所有段,并 flush WB 来实现. 之后将这些信号或上原来的信号即可.

\section{资源占用 \& 性能}
截图较多,关于资源利用的截图可以在 report 文件夹中查看.

更大的 cache 在综合的时候不知道为啥综合到一半就会失败,然后 vivado 会闪退,所以没法进行试验了.

以下是统计的资源利用的情况,我的 MATRIX 和 SORT 两个程序都使用默认的大小.

\begin{table}[H]
  \centering
  \begin{tabular}{c c c c c c}
    \toprule
     & Cache Size (WORD) & LUT & FF & MATRIX (ns, hitrate) & SORT (ns, hitrate)\\
    \midrule
    \makecell{LINE\_ADDR\_LEN = 2 \\ SET\_ADDR\_LEN = 3 \\ WAY\_CNT = 4} & 128 & 2360 & 5042 & \makecell{1335980 \\ 45.4\%} & \makecell{113260\\ 86.9\%} \\
    \midrule
    \makecell{LINE\_ADDR\_LEN = 3 \\ SET\_ADDR\_LEN = 3 \\ WAY\_CNT = 2} & 128 & 2006 & 5189 & \makecell{1320748 \\ 45.9\%} & \makecell{128308\\ 89.1\%} \\
    \midrule
    \makecell{LINE\_ADDR\_LEN = 3 \\ SET\_ADDR\_LEN = 3 \\ WAY\_CNT = 3} & 192 & 3018 & 7332 & \makecell{1317292  \\ 46.4\%} & \makecell{127244\\ 89.2\%}\\
    \midrule
    \makecell{LINE\_ADDR\_LEN = 3 \\ SET\_ADDR\_LEN = 3 \\ WAY\_CNT = 4} & 256 & 4355 & 9494 & \makecell{645380 \\ 82.2\%} & \makecell{87544\\ 96.4\%} \\
    \midrule
    \makecell{LINE\_ADDR\_LEN = 4 \\ SET\_ADDR\_LEN = 3 \\ WAY\_CNT = 2} & 256 & 4035 & 10044 & \makecell{499628 \\ 90.0\%} & \makecell{77032\\ 98.2\%} \\
    \midrule
    \makecell{LINE\_ADDR\_LEN = 4 \\ SET\_ADDR\_LEN = 3 \\ WAY\_CNT = 3} & 384 & 6355 & 14225 & \makecell{486456 \\ 90.7\%} & \makecell{70336\\ 99.3\%} \\
    \midrule
    \makecell{LINE\_ADDR\_LEN = 4 \\ SET\_ADDR\_LEN = 3 \\ WAY\_CNT = 4} & 512 & 9578 & 18430 & \makecell{269669 \\ 99.4\%} & \makecell{70336\\ 99.3\%} \\
    \midrule
    \makecell{LINE\_ADDR\_LEN = 3 \\ SET\_ADDR\_LEN = 4 \\ WAY\_CNT = 2} & 256 & 3918 & 9400 & \makecell{561572 \\ 86.6\%} & \makecell{82616\\ 97.2\%}\\
    \midrule
    \makecell{LINE\_ADDR\_LEN = 3 \\ SET\_ADDR\_LEN = 4 \\ WAY\_CNT = 3} & 384 & 6118 & 13676 & \makecell{502028 \\ 89.8\%} & \makecell{79512\\ 98.6\%}\\
    \midrule
    \makecell{LINE\_ADDR\_LEN = 3 \\ SET\_ADDR\_LEN = 4 \\ WAY\_CNT = 4} & 512 & 7410 & 17973 & \makecell{288416 \\ 98.6\%} & \makecell{79512\\ 98.6\%}\\
    \bottomrule
  \end{tabular}
  \caption{资源占用和性能表}
  \label{table:util_perf}
\end{table}

\section{分析}

\subsection{替换策略}

我自己选取了一些情况对替换策略进行实验,LRU 和 FIFO 差别不大,LRU 小优。但正如我在 \ref{impl:lru} 中所讲的,对于我的实现方式而言,两者所占用的资源是几乎完全一样的,我没必要针对 LRU 和 FIFO 进行资源上的权衡,所以我直接选取 LRU 作为我的替换策略。

\subsection{最佳参数和分析}

分析一下表格 \ref{table:util_perf},我们可以发现:

\begin{itemize}
  \item
    对于矩阵程序,我们可以发现,当 LINE\_ADDR\_LEN 或 WAY\_CNT 较小时,其命中率和效率明显低于其它情况,同时,每次 WAY\_CNT 从 3 到 4,都会带来性能的爆发.

    分析其原因,应当是矩阵乘法具有较强的空间局部性,因此,若 LINE\_ADDR\_LEN 较大,每次将内存中的数据替换出来时,都能替换出来很大一批数据,并且这些数据几乎都会在紧接下来的计算中被用到.

    同时,由于矩阵乘法设计到同列的连续读取,这些同列的数据虽然不再同一个 LINE 中,但它们的 SET\_ADDR 相同,因此会进入同一个组中. 当 WAY\_CNT 较大时,它们能够有效地同时存在于缓存中,而不至于被换来换去.
  \item
    对于排序程序而言,其并不具有很强的局部性,WAY\_CNT 增大而带来的性能提高并不明显,读取相对来说更加随机(但也具有一定的局部性),因此其命中率、运行时间随大小的关系比较平滑.
\end{itemize}

由如上分析,我们得出结论:

\begin{itemize}
  \item
    对于矩阵程序,其 cache 大小增大(具体来说是 LINE 和 WAY 的增大,能大幅度的提高性能,是比较值得的. 权衡资源占用后,我们可以选择 (LINE\_ADDR\_LEN = 3, SET\_ADDR\_LEN = 4, WAY\_CNT = 4) 的参数,此时命中率比较高,时间比较少(与最佳的一个相差不大),同时所使用的 LUT 少了很多.
  \item
    对于排序程序,其 cache 大小和 WAY 增大并不具有非常强的作用,推荐使用 (LINE\_ADDR\_LEN = 4, SET\_ADDR\_LEN = 3, WAY\_CNT = 2) 的情况,此时资源占用相对较少,性能也非常不错。
\end{itemize}

\end{document}
