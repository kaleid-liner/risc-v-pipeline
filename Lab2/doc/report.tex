\documentclass{article}

\usepackage{xeCJK}
\usepackage{fancyhdr}
\usepackage{extramarks}
\usepackage{amsmath}
\usepackage{amsthm}
\usepackage{amsfonts}
\usepackage{tikz}
\usepackage[plain]{algorithm}
\usepackage{algpseudocode}
\usepackage{enumerate}
\usepackage{xcolor, minted}
\usepackage[inline]{enumitem}
\usepackage{booktabs}
\usemintedstyle{colorful}
\definecolor{bg}{rgb}{0.95, 0.95, 0.95}
\definecolor{lk}{rgb}{0.5, 0.1, 0.1}
\setminted{bgcolor=bg,
           numbers=right,
           samepage}
\setmonofont{Droid Sans Mono for Powerline}

\usetikzlibrary{automata,positioning}

\usepackage[
  colorlinks,
  urlcolor=lk,
  linkcolor=black,
  citecolor=red,
  ]{hyperref}

\usepackage{tocloft}
\renewcommand\cfttoctitlefont{\huge\bfseries}
\renewcommand\cftsecfont{\Large\bfseries}
\renewcommand\cftsubsecfont{\large}
\renewcommand\cftsubsubsecfont{\large}
\setlength\cftbeforesecskip{20pt}
\setlength\cftbeforesubsecskip{10pt}
\setlength\cftbeforesubsubsecskip{8pt}

\setlength\parskip{0.8em}
\setlength\lineskip{0.5em}


%
% Basic Document Settings
%
\setmainfont[ItalicFont=Times New Roman Italic]{Arial}
\setCJKmainfont[ItalicFont=STKaiti, BoldFont=STHeiti]{STXihei}

\topmargin=-0.45in
\evensidemargin=0in
\oddsidemargin=0in
\textwidth=6.5in
\textheight=9.0in
\headsep=0.25in

\linespread{1.1}

\pagestyle{fancy}
\lhead{\reportAuthorName}
\chead{\reportMainTitle}
\rhead{\leftmark}
\cfoot{\thepage}

\renewcommand\headrulewidth{0.4pt}
\renewcommand\footrulewidth{0.4pt}

\setlength\parindent{0pt}

%
% Create Problem Sections
%


%
% Homework Details
%   - Title
%   - Due date
%   - Class
%   - Section/Time
%   - Instructor
%   - Author
%

\newcommand{\reportMainTitle}{Implementation of RISC-V 5-Stage Pipeline CPU}
\newcommand{\reportTitle}{Lab\ \#2}
\newcommand{\reportDueTime}{May 3, 2020 at 23:59pm}
\newcommand{\reportClass}{Computer Architecture}
\newcommand{\reportClassTime}{}
\newcommand{\reportClassInstructor}{周学海}
\newcommand{\reportAuthorName}{魏剑宇}
\newcommand{\reportStudentNo}{PB17111586}

%
% Title Page
%

\title{
  \vspace{2in}
  \textbf{\reportMainTitle}\\
  \vspace{0.2in}
  \Large\textit{\reportClass:\ \reportTitle}\\
  \vspace{0.1in}
  \normalsize\vspace{0.1in}\small{Due\ on\ \reportDueTime}\\
  \vspace{3in}
}

\author{\reportAuthorName \\ \reportStudentNo}
\date{April 30, 2020}


%
% Various Helper Commands
%

% Useful for algorithms
\newcommand{\alg}[1]{\textsc{\bfseries \footnotesize #1}}

% For derivatives
\newcommand{\deriv}[1]{\frac{\mathrm{d}}{\mathrm{d}x} (#1)}

% For partial derivatives
\newcommand{\pderiv}[2]{\frac{\partial}{\partial #1} (#2)}

% Integral dx
\newcommand{\dx}{\mathrm{d}x}

% Alias for the Solution section header

% Probability commands: Expectation, Variance, Covariance, Bias
\newcommand{\E}{\mathrm{E}}
\newcommand{\Var}{\mathrm{Var}}
\newcommand{\Cov}{\mathrm{Cov}}
\newcommand{\Bias}{\mathrm{Bias}}

\begin{document}

\maketitle

\pagebreak

\tableofcontents\label{toc}

\pagebreak

\newpage

\section{实验目的}

\begin{itemize}
  \item
    \textbf{Stage 1}: 实现一个基本的流水线 CPU,包含一些基本的指令
  \item
    \textbf{Stage 2}: 实现 RISC-V 的大部分用户级指令,并完成流水线的 Hazard 模块,解决数据相关、控制相关等问题
  \item
    \textbf{Stage 3}: 实现 CSR 模块和 CSR 相关的指令
\end{itemize}

\section{实验环境和工具}
\begin{itemize}
  \item
    \textbf{Host OS}: macOS Mojave 10.14.6
  \item
    \textbf{Guest OS}: Ubuntu Bionic Beaver
  \item
    \textbf{VSCode}: 1.44.2
  \item
    \textbf{Vivado}: HL WebPack Edition
  \item
    \textbf{Git}: 2.20.1 (Apple Git-117)
\end{itemize}
本次实验 Simulation 部分在 Ubuntu 虚拟机中完成 (Vivado 没有提供 macOS 版本),代码编写在 mac 上使用 VSCode 完成.

我的做法是在虚拟机 vivado 创建项目时,将源码的文件夹添加进去的时候,不拷贝,而是映射到项目中。然后在 mac 上通过共享文件夹将源码映射到 ubuntu 上,这样,源码编辑和 Git 版本控制都可以在 mac 上完成,只需在仿真的时候切换到 Ubuntu。整个一套 workflow 还是非常流畅舒适的。
\section{实验过程}
\subsection{Stage 1}
原本我是打算跳过 stage 1 直接编写 stage 2,但想到加入 Hazard 模块后,代码会比较复杂,如果在 ALU 和一些简单的数据通路处出现了问题,可能会比较难 Debug,所以还是先做了 Stage 1.

下面在介绍的过程中,我并非描述了我所有编写的代码。例如 BranchDecision 以及 ALU 中的一部分代码,逻辑都十分简单,或者之间有共同性,都介绍的话会比较冗余.
\subsubsection{Hazard}
为了先消除数据相关的影响,我令代码每运行一条指令,插入三个周期的气泡,完成这一步的代码如下. 插入气泡的方式是令 bubbleF 并 flushD.
\begin{minted}{verilog}
always @(posedge CPU_CLK) begin
    debug_hazard_cnt <= debug_hazard_cnt + 1;
end

initial begin
    debug_hazard_cnt = 0;
end

assign bubbleF = debug_hazard_cnt[0] | debug_hazard_cnt[1];
assign flushD = bubbleF | debug_flushD;
\end{minted}
\subsubsection{ALU}
ALU 部分实现 \mintinline{text}{Parameters.v} 中几个操作. 这个只需要使用一个 case 语句就可以完成了,唯一需要注意的是 SRA 和 SLT 操作。这两个操作涉及到补码. 在实现 SRA 使,我为了图方便就直接使用 \mintinline{text}{>>>} 操作符了. BranchDecision 中同样要注意补码的问题.
\begin{minted}{verilog}
wire signed [31:0] signed_op1 = op1;
always @(*) begin
    case (ALU_func)
        // ...
        `SRA   : ALU_out = signed_op1 >>> op2[4:0];
        // ...
        `SLT   : ALU_out = (op1 < op2)
                           & (op1[31] | ~op2[31])) | (op1[31] & ~op2[31]);
        // ...
    endcase
end
\end{minted}
\subsubsection{ControllerDecoder}
这可以说是阶段一整个 CPU 最重要的一部分了。在下面我有选择性地介绍一下其中一些信号的生成。

\begin{itemize}
  \item
    \textbf{op2\_src, alu\_src2}: op2 只有对于 R 类型的指令才来源于寄存器,其它情况下都来源于立即数. 对于 alu\_src2 由于我在 ALU 计算时进行了截取(因为移位指令要求后一个小于 32),所以没有生成需要 reg2 地址的情况.
    \begin{minted}{verilog}
assign op2_src = imm_type == `RTYPE ? 0 : 1;
assign alu_src2 = op2_src ? 2'b10 : 2'b00;
    \end{minted}
  \item
    \textbf{load\_npc}: 对于 JAL 指令和 JALR 指令,最终 WB 的数应该是 NPC,而不是 ALU 的计算结果.
    \begin{minted}{verilog}
assign load_npc = opcode == `OP_JAL || opcode == `OP_JALR;
    \end{minted}
  \item
    \textbf{ALU\_func}: ALU\_func 只需要通过一些 if 和 case 语句,针对 opcode 和 funct3 进行分情况讨论就行了. 需要注意,如 AUIPC, Store, Load, JALR 等指令都是需要进行 ADD 操作.
  \item
    \textbf{cache\_write\_en}: 只有对于 Store 类指令才不为 0,并且根据类型,其每一个 bit 表示的可写性不同. 例如,对于 sh 指令,则只有低两个字节可写.
    \begin{minted}{verilog}
if (opcode == `OP_STORE)
    case (funct3)
        3'b000 : cache_write_en = 4'b0001;
        3'b001 : cache_write_en = 4'b0011;
        3'b010 : cache_write_en = 4'b1111;
        default: cache_write_en = 4'b0000;
    endcase
else
    cache_write_en = 0;
    \end{minted}
  \item
    \textbf{imm\_type, src\_reg\_en, reg\_write\_en}: 这些都由 opcode 唯一决定,并且比较简单,我就放在一起了. 需要注意,src\_reg\_en 和 reg\_write\_en 在后面的数据相关中发挥了很重要的作用. 另外,需要特别注意,JAL 和 JALR 的 reg\_write\_en 也为 1.
    \begin{minted}{verilog}
`OP_JAL    : begin
    imm_type = `JTYPE;
    src_reg_en = 2'b00;
    reg_write_en = 1;
end
`OP_JALR   : begin
    imm_type = `ITYPE;
    src_reg_en = 2'b10;
    reg_write_en = 1;
end
    \end{minted}
\end{itemize}

\subsubsection{DataExtend}
框架代码给出了一个不支持非对齐的内存实现,我也就没有考虑这方面的问题. 在 DataExtend 中,需要通过 addr 判断从哪个地方开始读取,另外还需要对读取的数据进行扩展. 如下,是读取一个字节的情况
\begin{minted}{verilog}
`LB     : begin
    case (addr)
        2'b00: data_before_extend[7:0] = data[7:0];
        2'b01: data_before_extend[7:0] = data[15:8];
        2'b10: data_before_extend[7:0] = data[23:16];
        2'b11: data_before_extend[7:0] = data[31:24];
    endcase
    dealt_data = {{24{data_before_extend[7]}}, data_before_extend[7:0]};
end
\end{minted}
\subsubsection{NPC Generator}
在 lab1 中我们讨论过,各个跳转或分支指令是有优先级的,在 EX 段完成跳转的优先级应该高于在 ID 段完成跳转指令的优先级.
\begin{minted}{verilog}
always @(*) begin
  if (jalr)
      NPC = jalr_target;
  else if (br)
      NPC = br_target;
  else if (jal)
      NPC = jal_target;
  else
      NPC = PC;
end
\end{minted}
\subsection{Stage 2}
在阶段一我就已经把另外那些指令也实现了,这里也不讨论了. 在这节我将介绍一下 Hazard 模块的实现.
\subsubsection{Data Hazard}
阅读了框架代码后,我发现助教的思路和我的思路不同. 框架是对于两条相关的指令,考虑对于前一条指令,如果那条指令与后面的某一条指令相关,就需要插入 bubble. 而我是考虑对于后面一条指令,是否与前面的指令相关,如果是,则插入 bubble.

两种思路都是可以的. 在我的思路下,我需要更改 Hazard 的接口,wb\_select 应该来自于 MEM 而不是 EX,并且 reg1\_srcD, reg2\_srcD, reg\_dstE 将不会被用到.

对于 Data Hazard,有三种情况:
\begin{enumerate}
  \item
    EX 用到的 reg 来自于 MEM,即 reg2 用到了,不是 0,MEM 有写,并且两者相同. 这种情况又分为两种情况:该条指令是通过 ALU 在 EX 计算得出结果或者在 MEM 加载得到结果. 这两种情况可以通过 wb\_select 区分. 如果 EX 就能得到结果,直接 forward 即可,如果 MEM 才能得到结果,就需要插入 bubble.
    \begin{minted}{verilog}
if (src_reg_en[0] && reg2_srcE != 5'b0
    && reg_write_en_MEM && reg2_srcE == reg_dstM) begin
    if (!wb_select) begin
        op2_sel = alu_src2 == 2'b00 ? 2'b00 : 2'b11;
        reg2_sel = 2'b00;
        reg2_bubbleF = 0;
        reg2_bubbleD = 0;
        reg2_bubbleE = 0;
        reg2_flushM = 0;
    end else begin
        op2_sel = alu_src2 == 2'b00 ? 2'b01 : 2'b11;
        reg2_sel = 2'b01;
        reg2_bubbleF = 1;
        reg2_bubbleD = 1;
        reg2_bubbleE = 1;
        reg2_flushM = 1;
    end
end
    \end{minted}
    注意,在 EX 段插入 bubble 是通过 bubbleF,bubbleD,bubbleE,flushM 来实现的.
  \item
    EX 用到的 reg 来自于 WB, 此时同样直接转发,不需要 bubble.
    \begin{minted}{verilog}
else if (src_reg_en[0] && reg2_srcE != 5'b0
         && reg_write_en_WB && reg2_srcE == reg_dstW) begin
    op2_sel = alu_src2 == 2'b00 ? 2'b01 : 2'b11;
    reg2_sel = 2'b01;
    reg2_bubbleF = 0;
    reg2_bubbleD = 0;
    reg2_bubbleE = 0;
    reg2_flushM = 0;
end
    \end{minted}
  \item
    无数据相关,则不需要转发也不需要 bubble.
    \begin{minted}{verilog}
else begin
    op2_sel = 2'b11;
    reg2_sel = 2'b11;
    reg2_bubbleF = 0;
    reg2_bubbleD = 0;
    reg2_bubbleE = 0;
    reg2_flushM = 0;
end
    \end{minted}
\end{enumerate}
reg1 的情况类似.
\subsubsection{Control Hazard}
对于 EX 段完成的跳转指令,需要冲刷流水线两个周期. 对于 ID 段完成的跳转指令,需要冲刷流水线一个周期. 我直接冲刷流水线,就没必要插入 bubble 了.
\begin{minted}{verilog}
always @(*) begin
    if (jalr) begin
        jump_flushD = 1;
        jump_flushE = 1;
    end else if (br) begin
        jump_flushD = 1;
        jump_flushE = 1;
    end else if (jal) begin
        jump_flushD = 1;
        jump_flushE = 0;
    end else begin
        jump_flushD = 0;
        jump_flushE = 0;
    end
end
\end{minted}
可以发现,上面我都是通过 jump, reg1, reg2 来表示由于各个原因引发的 bubble 或 flush. 总的我只需要把他们都或一起就行了,例如:
\begin{minted}{verilog}
flushM = reg1_flushM | reg2_flushM | rst;
\end{minted}
\subsection{Stage 3}
这一步为了不引进更多数据依赖(偷懒省事),我没有按照一般的方法,在 ID 段读取 CSR,在 WB 段写入 CSR,而是直接全部都在 EX 段完成. 这样,这样对 CSR 的读取不需要 bypass 或插入 bubble,只需要对 CSR 用到的通用寄存器进行 bypass. 而这一步不需要更改 Hazard 模块的代码,只需要正确生成 CSR 的 src\_reg\_en 和 reg\_write\_en 即可.

\subsubsection{CSR}
CSR 模块中,我塞进去了一堆寄存器和一个 ALU. 如下:
\begin{minted}{verilog}
// ...
always @(*) begin
    case (op)
        `CSRRW : dealt_data = in_data;
        `CSRRC : dealt_data = out_data & ~in_data;
        `CSRRS : dealt_data = in_data | out_data;
        default: dealt_data = in_data;
    endcase
end

always @(posedge clk or posedge rst) begin
    if (rst)
        for (i = 0; i < 32; i = i + 1)
            reg_file[i][31:0] <= 32'b0;
    else if (write_en)
        reg_file[dealt_addr] <= dealt_data;
end
\end{minted}
注意 CSRRC 应当是一个与非操作.
\subsubsection{Controller Decoder}
另外还需要 CSR 所需要的控制信号.
\begin{itemize}
  \item
    \textbf{csr\_src}: 表示来自 zimm 或者 reg1.
    \begin{minted}{verilog}
assign csr_src = funct3[2];
    \end{minted}
    我发现 funct3 的高一位 bit 表示是否是立即数.
  \item
    \textbf{load\_csr}: 和 load\_npc 类似,表示 EX 段产生的结果来自 CSR 模块而不是 ALU 或 npc.
    \begin{minted}{verilog}
assign load_csr = opcode == `OP_CSR;
    \end{minted}
  \item
    \textbf{csr\_read\_en, csr\_write\_en}: 表示 csr 的可写或者可读. 当然我这里写或者读没有副作用,所以 csr\_read\_en 也就没有用到.
    \begin{minted}{verilog}
assign csr_write_en = opcode == `OP_CSR
                      && (rs1 != 5'd0 || funct3[1:0] == 2'b01);
assign csr_read_en = opcode == `OP_CSR && rd != 5'd0;
    \end{minted}
  \item
    \textbf{src\_reg\_en, csr\_write\_en}: 需要更新一下这两个控制信号:
    \begin{minted}{verilog}
`OP_CSR    : begin
    imm_type = `ITYPE;
    src_reg_en = csr_src ? 2'b00 : 2'b10;
    reg_write_en = 1;
end
    \end{minted}
\end{itemize}
\section{实验总结}
\subsection{Bug Fix}
在我完成此实验的过程中,coding:debug 的比例大概在 1:2 吧. 这里,我参照着 commit message,尽量回忆一些我踩的坑吧.
\begin{itemize}
  \item
    转发时需要考虑优先级的问题. 如果 MEM 和 WB 段都有产生的结果,需要使用 MEM 而不是 WB 的,因为 MEM 产生的结果是相对 WB 的后一条指令,才是最新结果.
  \item
    对有符号数的移位操作,除了通过数字逻辑实现之外,另一种方式是让 vivado 帮我们做,使用 \mintinline{text}{>>>} 操作符. 但此操作符要求 op1 是 signed.
  \item
    对于 store 指令,虽然它既用到了 reg1,也用到了 reg2,但对于 reg2,是写入的数据,而不会在 ALU 中用到. 我们需要转发 reg2 而不需要转发 op2.
\end{itemize}
其它的 bug 是一些不重要的东西,比如 typo 或者意识模糊的时候写的代码,就不再这里讨论了.
\subsection{收获 \& 分析}
这次实验进一步加强(?)了自己的 verilog 能力,当然主要是对流水线有了更深的理解. 像这样实现一次后,很多之前没有搞明白的事情现在就搞清楚了.

这次实验我花了三天时间,每天大概 5 个小时. 第一天 stage1,第二天 stage2,第三天 stage3. 总的来说,在助教搭好的框架上写,还是相对比较容易的. 毕竟只剩下完形填空的活了,也不需要自己设计.
\end{document}
