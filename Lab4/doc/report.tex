\documentclass{article}

\usepackage{xeCJK}
\usepackage{fancyhdr}
\usepackage{extramarks}
\usepackage{amsmath}
\usepackage{amsthm}
\usepackage{amsfonts}
\usepackage{tikz}
\usepackage[plain]{algorithm}
\usepackage{algpseudocode}
\usepackage{enumerate}
\usepackage{xcolor, minted}
\usepackage[inline]{enumitem}
\usepackage{booktabs}
\usepackage{makecell}
\usepackage{multirow}
\usemintedstyle{colorful}
\definecolor{bg}{rgb}{0.95, 0.95, 0.95}
\definecolor{lk}{rgb}{0.5, 0.1, 0.1}
\setminted{bgcolor=bg,
           numbers=right,
           samepage}
\setmonofont{Droid Sans Mono for Powerline}

\usetikzlibrary{automata,positioning}

\usepackage[
  colorlinks,
  urlcolor=lk,
  linkcolor=black,
  citecolor=red,
  ]{hyperref}

\usepackage{tocloft}
\renewcommand\cfttoctitlefont{\huge\bfseries}
\renewcommand\cftsecfont{\Large\bfseries}
\renewcommand\cftsubsecfont{\large}
\renewcommand\cftsubsubsecfont{\large}
\setlength\cftbeforesecskip{20pt}
\setlength\cftbeforesubsecskip{10pt}
\setlength\cftbeforesubsubsecskip{8pt}

\setlength\parskip{0.8em}
\setlength\lineskip{0.5em}


%
% Basic Document Settings
%
\setmainfont[ItalicFont=Times New Roman Italic]{Arial}
\setCJKmainfont[ItalicFont=STKaiti, BoldFont=STHeiti]{STXihei}

\topmargin=-0.45in
\evensidemargin=0in
\oddsidemargin=0in
\textwidth=6.5in
\textheight=9.0in
\headsep=0.25in

\linespread{1.1}

\pagestyle{fancy}
\lhead{\reportAuthorName}
\chead{\reportMainTitle}
\rhead{\leftmark}
\cfoot{\thepage}

\renewcommand\headrulewidth{0.4pt}
\renewcommand\footrulewidth{0.4pt}

\setlength\parindent{0pt}

%
% Create Problem Sections
%


%
% Homework Details
%   - Title
%   - Due date
%   - Class
%   - Section/Time
%   - Instructor
%   - Author
%

\newcommand{\reportMainTitle}{Implementation of Branch Decision}
\newcommand{\reportTitle}{Lab\ \#4}
\newcommand{\reportDueTime}{May 31, 2020 at 23:59pm}
\newcommand{\reportClass}{Computer Architecture}
\newcommand{\reportClassTime}{}
\newcommand{\reportClassInstructor}{周学海}
\newcommand{\reportAuthorName}{魏剑宇}
\newcommand{\reportStudentNo}{PB17111586}

%
% Title Page
%

\title{
  \vspace{2in}
  \textbf{\reportMainTitle}\\
  \vspace{0.2in}
  \Large\textit{\reportClass:\ \reportTitle}\\
  \vspace{0.1in}
  \normalsize\vspace{0.1in}\small{Due\ on\ \reportDueTime}\\
  \vspace{3in}
}

\author{\reportAuthorName \\ \reportStudentNo}
\date{May 31, 2020}


%
% Various Helper Commands
%

% Useful for algorithms
\newcommand{\alg}[1]{\textsc{\bfseries \footnotesize #1}}

% For derivatives
\newcommand{\deriv}[1]{\frac{\mathrm{d}}{\mathrm{d}x} (#1)}

% For partial derivatives
\newcommand{\pderiv}[2]{\frac{\partial}{\partial #1} (#2)}

% Integral dx
\newcommand{\dx}{\mathrm{d}x}

% Alias for the Solution section header

% Probability commands: Expectation, Variance, Covariance, Bias
\newcommand{\E}{\mathrm{E}}
\newcommand{\Var}{\mathrm{Var}}
\newcommand{\Cov}{\mathrm{Cov}}
\newcommand{\Bias}{\mathrm{Bias}}

\begin{document}

\maketitle

\pagebreak

\tableofcontents\label{toc}

\pagebreak

\newpage

\section{实验环境和工具}
\begin{itemize}
  \item
    \textbf{Host OS}: macOS Mojave 10.14.6
  \item
    \textbf{Guest OS}: Ubuntu Bionic Beaver
  \item
    \textbf{VSCode}: 1.44.2
  \item
    \textbf{Vivado}: HL WebPack Edition
  \item
    \textbf{Git}: 2.20.1 (Apple Git-117)
\end{itemize}
本次实验 Simulation 部分在 Ubuntu 虚拟机中完成 (Vivado 没有提供 macOS 版本),代码编写在 mac 上使用 VSCode 完成.

我的做法是在虚拟机 vivado 创建项目时,将源码的文件夹添加进去的时候,不拷贝,而是映射到项目中。然后在 mac 上通过共享文件夹将源码映射到 ubuntu 上,这样,源码编辑和 Git 版本控制都可以在 mac 上完成,只需在仿真的时候切换到 Ubuntu。整个一套 workflow 还是非常流畅舒适的。

\section{Branch History Table}
下表中 target 表示既不在 BTB 中,也不是 PC\_IF+4,而是指令中决定的跳转地址。
\begin{table}[H]
  \centering
  \begin{tabular}{c c c c c c c}
    \toprule
    BTB & BHT & REAL & NPC\_PRED & flush & NPC\_REAL & BTB update \\
    \midrule
    Y   & Y   & Y    & BUF       & N     & BUF       & N          \\
    Y   & Y   & N    & BUF       & Y     & PC\_IF+4  & N          \\
    Y   & N   & Y    & PC\_IF+4  & Y     & BUF       & N          \\
    Y   & N   & N    & PC\_IF+4  & N     & PC\_IF+4  & N          \\
    N   & Y   & Y    & PC\_IF+4  & Y     & target    & Y          \\
    N   & Y   & N    & PC\_IF+4  & N     & PC\_IF+4  & N          \\
    N   & N   & Y    & PC\_IF+4  & Y     & target    & Y          \\
    N   & N   & N    & PC\_IF+4  & N     & PC\_IF+4  & N          \\
    \bottomrule
  \end{tabular}
  \caption{BHT 策略矩阵}
\end{table}

\section{性能测试 \& 分析}
下面在进行性能测试的时候,我都是截止到最后死循环的时候(而不是死循环前的那个遍历数组的循环).

例如,如下是矩阵乘法时结束的时间点:
\begin{figure}[H]
  \centering
  \includegraphics[width=10cm]{simulation.png}
\end{figure}

另外,对于 bht.S 和 btb.S 两个程序,由于最后没有死循环,直接截止到最后一个循环执行的时间即可.(之后指令是 XXX)

在测试的时候需要注意,其它几个程序都没问题,但对于 quicksort 需要保证内存是同一次生成的。因为内存是随机生成的,对其它的程序的运行流程没有影响,但对于排序会造成影响。

同时在测试的时候,我的 cache 使用的同一个参数 (WAY\_CNT=4, SET\_ADDR\_LEN=3, LINE\_ADDR\_LEN=3).

得到结果如下(矩阵乘法和排序皆使用默认参数):
\begin{table}[H]
  \centering
  \begin{tabular}{c c c c c c c c}
    \toprule
    程序名 & 分支策略 & 时间 (ns) & 加速比 & 正确次数 & 错误次数 & 总次数 & 正确率\\
    \midrule
    \multirow{3}{*}{matmul.s} & BHT & 613880 & 1.053 & 4342 & 282 & 4624 & 93.9\% \\
                              & BTB & 616016 & 1.049 & 4076 & 548 & 4624 & 88.1\% \\
                              & baseline & 646432 & 1 & 274 & 4350 & 4624 & 5.9\% \\
    \midrule
    \multirow{3}{*}{quicksort.s} & BHT & 159484 & 1.020 & 10475 & 1908 & 12383 & 84.6\% \\
                                 & BTB & 161692 & 1.006 & 8749 & 3634 & 12383 & 70.7\% \\
                                 & baseline & 162710 & 1 & 7871 & 4512 & 12383 & 63.6\% \\
    \midrule
    \multirow{3}{*}{btb.s} & BHT & 1260 & 1.616 & 98 & 3 & 101 & 97.0\% \\
                           & BTB & 1252 & 1.626 & 99 & 2 & 101 & 98.0\% \\
                           & baseline & 2036 & 1 & 1 & 100 & 101 & 1.0\% \\
    \midrule
    \multirow{3}{*}{bht.s} & BHT & 1472 & 1.454 & 95 & 15 & 120 & 79.2\% \\
                           & BTB & 1524 & 1.404 & 88 & 22 & 120 & 73.3\% \\
                           & baseline & 2140 & 1 & 11 & 99 & 120 & 9.2\% \\
    \bottomrule
  \end{tabular}
  \caption{性能测试}
  \label{tb:perf}
\end{table}
在上表 \ref{tb:perf} 中,baseline 代表“没有分支预测的情况”。但实际上,由于在我前面的实现中,我实现了静态的分支预测(永远预测不跳转),所以 baseline 代表的是一直预测不跳转的性能(也因此有预测正确的个数和不正确的个数).

观察上表我们可以发现:

\begin{enumerate}
  \item
    BHT 大部分情况下都稍优于单纯的 BTB 实现
\end{enumerate}

\subsection{分支收益和分支代价}

分析实验数据,btb.s 这一行 BHT 和 BTB 的对比能够很清晰地展现出预测所带来的收益。对于 BHT 和 BTB 两种实现策略,预测正确的次数差了 1,而时间差了 8ns。每一个时钟周期是 4ns,这刚好是两个时钟周期的时间。也就是说,预测错误的分支代价是 2.

理论上进行分析,每次错误的预测,都会使得实际到 EX 段才会得出正确的分支结果,这样我们需要 flush 调已经取入的错误指令(也就是下一个周期的 ID 和 EX),这样,我们会浪费掉两个流水线周期,和实验的结果相符。

总体的分支收益上,我们可以观察实验数据,两种预测策略在不同程序上都得到了一定加速比,且正确率也不错。因此而带来的分支收益是 $正确率 \times 总次数 \times 8ns$.

\subsection{BHT vs BTB}
我们可以发现,在快速排序和矩阵乘法中,BHT 都稍微优于 BTB. 只有在 btb.s 中 BTB 稍微更胜一筹。分析具体的原因,我们对比 btb.s 和 bht.s 这两个相对比较简单的程序,更容易分析。

观察这两个程序我们发现,bht.s 是一个双层循环,btb.s 是一个单层循环。在双层循环中,每一个内层循环的末尾都将是一个实际不跳转,内层循环的开头都将是一个实际跳转。如果使用 BTB 作为分支策略,那么每一次内层循环的末尾都将预测失败,且由于上一次内层循环的末尾预测跳转而实际不跳转,下一次开头时就会预测不跳转,但实际跳转。对于 bht 则没有这种情况



\end{document}
